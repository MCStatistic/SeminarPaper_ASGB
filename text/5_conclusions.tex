\documentclass[]{article}
\usepackage{lmodern}
\usepackage{amssymb,amsmath}
\usepackage{ifxetex,ifluatex}
\usepackage{fixltx2e} % provides \textsubscript
\ifnum 0\ifxetex 1\fi\ifluatex 1\fi=0 % if pdftex
  \usepackage[T1]{fontenc}
  \usepackage[utf8]{inputenc}
\else % if luatex or xelatex
  \ifxetex
    \usepackage{mathspec}
  \else
    \usepackage{fontspec}
  \fi
  \defaultfontfeatures{Ligatures=TeX,Scale=MatchLowercase}
\fi
% use upquote if available, for straight quotes in verbatim environments
\IfFileExists{upquote.sty}{\usepackage{upquote}}{}
% use microtype if available
\IfFileExists{microtype.sty}{%
\usepackage{microtype}
\UseMicrotypeSet[protrusion]{basicmath} % disable protrusion for tt fonts
}{}
\usepackage[margin=1in]{geometry}
\usepackage{hyperref}
\hypersetup{unicode=true,
            pdfborder={0 0 0},
            breaklinks=true}
\urlstyle{same}  % don't use monospace font for urls
\usepackage{graphicx,grffile}
\makeatletter
\def\maxwidth{\ifdim\Gin@nat@width>\linewidth\linewidth\else\Gin@nat@width\fi}
\def\maxheight{\ifdim\Gin@nat@height>\textheight\textheight\else\Gin@nat@height\fi}
\makeatother
% Scale images if necessary, so that they will not overflow the page
% margins by default, and it is still possible to overwrite the defaults
% using explicit options in \includegraphics[width, height, ...]{}
\setkeys{Gin}{width=\maxwidth,height=\maxheight,keepaspectratio}
\IfFileExists{parskip.sty}{%
\usepackage{parskip}
}{% else
\setlength{\parindent}{0pt}
\setlength{\parskip}{6pt plus 2pt minus 1pt}
}
\setlength{\emergencystretch}{3em}  % prevent overfull lines
\providecommand{\tightlist}{%
  \setlength{\itemsep}{0pt}\setlength{\parskip}{0pt}}
\setcounter{secnumdepth}{0}
% Redefines (sub)paragraphs to behave more like sections
\ifx\paragraph\undefined\else
\let\oldparagraph\paragraph
\renewcommand{\paragraph}[1]{\oldparagraph{#1}\mbox{}}
\fi
\ifx\subparagraph\undefined\else
\let\oldsubparagraph\subparagraph
\renewcommand{\subparagraph}[1]{\oldsubparagraph{#1}\mbox{}}
\fi

%%% Use protect on footnotes to avoid problems with footnotes in titles
\let\rmarkdownfootnote\footnote%
\def\footnote{\protect\rmarkdownfootnote}

%%% Change title format to be more compact
\usepackage{titling}

% Create subtitle command for use in maketitle
\newcommand{\subtitle}[1]{
  \posttitle{
    \begin{center}\large#1\end{center}
    }
}

\setlength{\droptitle}{-2em}

  \title{}
    \pretitle{\vspace{\droptitle}}
  \posttitle{}
    \author{}
    \preauthor{}\postauthor{}
    \date{}
    \predate{}\postdate{}
  

\begin{document}

This section will briefly summarize the results and then conclude the
study with a few words on limitations and future research.

With regard to our multinomial logistic models, it can be said that our
hypotheses were able to be mainly \emph{confirmed}. Let's start with the
economic hypothesis:

\begin{quote}
H1: The less economically fortunate (economic dissatisfaction,
unemployment, economic insecurity, living on welfare), the higher the
probability of supporting populist parties.
\end{quote}

The models estimated here mainly support this hypothesis. Unemployment,
economic insecurity and receiving welfare increase the chance to support
populists compared to support of established parties. Altough, the
impact of living on welfare seems not to be as great as the other two
factors mentioned and does not yield a significant effect in supporting
traditionalist populists. In addition, in Model 3, where the cultural
dimension has been added, there is a slight reduction of the effects.
Despite this, all significances remain stable and the estimates continue
to show the expected effects.

Concerning the cultural dimension, the results of the multinomial
logistic regression also support our hypotheses:

\begin{quote}
H2: The more culturally inclusive (values), the higher the probability
of supporting progressive populist parties.
\end{quote}

\begin{quote}
H3: The more culturally exclusive, the higher the probability of
supporting a traditionalist populist party.
\end{quote}

Inclusive values (high values for self-transcendence and openness and
lower anti-immigration sentiment) increases the probability of support
for progressive populists, but openness lacks a statistically
significant effect. In addition, exclusive values (low values for
self-transcendence and openness and a high value for anti-immigration),
increase the probability of supporting traditionalist populism.
Anti-immigration in particular seems very important here and stands out
clearly from the other effects. Additionally, openness values decrease
the chance of supporting traditionalist populist parties. Even in Model
4, which combines all predictors, all estimated effects remain stable.
The effects also show our previously suspected differentiation. The
economic dimension increases the likelihood of supporting populist
parties in general. The cultural dimension, on the other hand, shows
that it has diametrically opposed effects on the support of
traditionalist and progressive populists.

The last hypothesis concerning interaction effects stated as follows:

\begin{quote}
H4: When economic and cultural factors interact, the support for
populist parties is higher.
\end{quote}

The empirical evidence obtained through multinomial logistic regression
does not support this hypothesis. Two statistically significant effects
were found, however most interactions were not. An interaction between
\emph{Anti-Immigration Sentiment} and \emph{Economic Insecurity} implied
that the effect of the former on populist party support is \emph{weaker}
for those that are very economically insecure, suggesting the direct
opposite of what we expected. Future research should further investigate
what factors might cause this discrepancy. On the other hand, the other
interaction effect between \emph{Unemployment} and
\emph{Self-Enhancment} showed the expected direction and the influence
of self-enhancement values was \emph{stronger} for those that are
unemployed. However, given the overwhelming contradictory evidence, this
hypothesis has to be rejected.

\begin{table}[]
    \centering
    \caption{Summary of Results}
    \label{summaryy}
    \resizebox{\textwidth}{!}{%
        \begin{tabular}{@{}lccl@{}}
            \toprule
            \multicolumn{1}{c}{\textbf{Hypothesis}} & 
            \textbf{Confirmed} & 
            \textbf{\begin{tabular}[c]{@{}c@{}}Indications/\\  Tendencies\end{tabular}} &
            \multicolumn{1}{c}{\textbf{Restrictions}} \\ 
            \midrule
            \textit{\begin{tabular}[c]{@{}l@{}}\textbf{H1:} The less economically fortunate \\(economic dissatisfaction, unemployment, \\ economic insecurity, living on welfare), \\the higher the probability of supporting \\ anti-establishment parties.\end{tabular}}                                  & Yes  & +                                                                          & \begin{tabular}[c]{@{}l@{}}Weak Effects \\ 
            Welfare not signicant for \\
            Traditionalist Populists\end{tabular}                           \\ \midrule
            
\textit{\begin{tabular}[c]{@{}l@{}}\textbf{H2:} The more culturally inclusive \\(values), the higher the probability of \\ supporting progressive populist parties.\end{tabular}}               & Yes                                                               & +                                                                          & \begin{tabular}[c]{@{}l@{}}Weak Effects \\ 
            Openness not signicant for \\
            Progressive Populists. \end{tabular}                                                                                      \\ \midrule
            \textit{\begin{tabular}[c]{@{}l@{}}\textbf{H3:} The more culturally exclusive, \\the higher the probability of supporting \\a traditionalist populist party.\end{tabular}}             & Yes                                                                & +                                                                          & \begin{tabular}[c]{@{}l@{}}Weak Effects. \end{tabular} \\ \midrule
            \textit{\begin{tabular}[c]{@{}l@{}}\textbf{H4:} When economic and cultural factors interact, \\ the support for populist parties is higher.\end{tabular}}               & No                                                                & \textbf{-}                                                                 & \begin{tabular}[c]{@{}l@{}}No significant effects \\
            except for two \\
            inconclusive evidence\end{tabular}                                                            \\ \bottomrule    \\[-1em]
            \multicolumn{4}{l}{%
                \begin{minipage}{19cm}%
                    \flushright
                    \scriptsize A "+" sign indicates at least some empirical evidence in favor of the hypothesis and a "-" sign stands for contradictory empirical evidence.%
                \end{minipage}%
            }
        \end{tabular}
}
\end{table}

It should also be stressed that while all the effects found are
significant, they are rather weak, especially when it comes to the
cultural dimension. In addition, almost all control effects remain
significant in all models. In the academic literature, supporters of
populist parties are often associated with features such as older age
and lower education. Some minor surprises were found here: in terms of
educational attainment, more education decreases the support for
traditional populists, whereas the same isn't true for supporters of
progressive populists where higher education levels are associated with
higher support. Further, when it comes to age, the models suggest that
younger age is associated with higher support for progressive as well as
traditional populists.

The goal of this paper was to bring some clarity on the confusion
between different variants of populism that all too often have neglected
this multidimensionality. Further research might be able to build upon
our conceptualization and give more attention to the different variants
of populism, so as to not conflate the distinct explanatory frameworks
that come along with them.


\end{document}
